\documentclass[../main.tex]{subfiles}

\begin{document}
It has been realized over the past decade that the black hole(BH) at the centre of a galaxy bulge is no mere ornament but may play a major role in galaxy evolution. The process by which this occurs is known as AGN(Active Galactic Nucleus) feedback and it takes place through an interaction between the energy and radiation generated by accretion onto the massive BH and the gas in the host galaxy. The possibility arises where the intense flux of photons and particles produced by the AGN sweeps the galaxy bulge clean of interstellar gas, terminates star formation, and through lack of fuel for acrretion, terminates the AGN. However, the ratio of the size of the BH to host galaxy is tiny and similar to coin in comparison to the Earth. In contrary to this comparison, AGN have a potent effect on galaxy formation and evolution even the large-scale structure. For example, the energy released to build a BH with mass $M_{BH}=10^{8} \mathbf{M}_{\odot}$ would correspond to $E_{BH} \approx 0.1M_{BH}c^{2}$. This total accretion energy is two-to-three order of magnitude higher than the binding energy of the galaxy bulge in which this BH is reside, and is comparable to, even higher than, the thermal energy of the gas in the dark matter halo in which this galaxy resides. So the study on AGN feedback is extremely important for galaxy even cosmology.

Theoretical models of galaxy formation and evolution have found it necessary to implement AGN feedback processes, during which AGN activity injects energy into the gas in the larger scale environment in order to produce the observational evidence. There are two sorts of AGN feedback divided according to mechanism. The first one is radio mode feedback also known as maintenance mode or kinetic mode, this mode of feedback is thought to be most efficient in the most massive halos, at late times and during periods of low BH accretion rates. The objects that are responsible for this type of feedback are likely to be low-excitation radio AGN which has low accretion rate and is always companied by radio lobes(\cite{churazov2005supermassive};\cite{bower2006breaking};\cite{mccarthy2011gas}). This method of feedback is mainly use to explain the cut-off at the bright of the galaxy luminosity function in cluster environment(\cite{bower2006breaking};\cite{croton2006many};\cite{somerville2008semi}). In contrast, a more catastrophic form of interaction between AGN and their host galaxies is proposed during periods of rapid accretion known as quasar mode or radiative mode. The objects are predicted to be responsible for this type of feedback are the radiatively-efficient AGN which has large accretion rate. Theoretical models that invoke this form of feedback typically require $\approx 0.005-0.15$ of the accretion energy to couple to the cold gas within the host galaxy and to expel this gas through outflows which ultimately results in the shut-down of future BH growth or star formation(\cite{benson2003shapes};\cite{hopkins2006normalization};\cite{debuhr2012galaxy}). In addition, analytical models have used the idea of galaxy-scale outflows initially launched by AGN, to explain $M_{BH}-M_{bulge}$ relationship(\cite{fabian1999obscured};\cite{granato2004physical};\cite{king2011large};\cite{faucher2012physics}). While this form of feedback has been predicted to be effective at producing high mass outflow rates and quench star formation. In addition to the potential affects of AGN feedback described above, AGN-driven outflows may be required to blow the gas in their host galaxy out to explain the chemical enrichment of intercluster medium(ICM) and circumgalactic medium(CGM)(\cite{borgani2008chemical};\cite{wiersma2009chemical};\cite{fabjan2010simulating};\cite{ciotti2010feedback}). It has also been prompted that, in some cases, these outflows could induce positive feedback which trigger star formation by inducing pressure in cold gas reservoirs(\cite{nayakshin2012quasar};\cite{ishibashi2012active};\cite{silk2013unleashing}). So the theory of galaxy formation and evolution strongly depends on AGN feedback models. However, there are still many problems to be solved, it is not well established how the accretion energy couples to the gas to drive outflows and the impact of outflow on a much larger physical scale. Observations are required to constrain the details of how, when and where these processes actually occur.

While there is no doubt that star formation processes also have ability to drive galaxy-wide outflow, it is believed that AGN activity is required to drive the highest velocity outflows and are particularly important for the evolution of most massive galaxies(\cite{benson2003shapes};\cite{mccarthy2011gas};\cite{harrison2018agn}). AGN-driven outflows are initially launched from the accretion disk or dusty torus surrounding the BH, either in the form of radio jet or radiatively-driven wind. These jet or winds then couple to the surrounding gas and dust resulting in large-scale outflows. Usually, spatially resolved spectroscopy is used as a direct way to search for and characterize outflows. Such observations have identified outflows in ionized, atomic and molecular gas(\cite{nesvadba2008evidence}). A diagnostic that is commonly used to search for outflowing ionized gas is broad, asymmetric and high-velocity OIII$\lambda5007$ emission line. This is the good tracer of kinematics in the narrow-line region(NLR) of AGN and people have used it to study kinematics in hundreds to ten of thousands of AGN(\cite{wang2011hierarchical};\cite{mullaney2013narrow}) to constrain the ubiquity of these outflow features and study them as a function of key AGN properties. 

These studies have used only one-dimensional spectra and therefore provide no insight on the spatial extent or structure of the outflows. Even for long-slit spectroscopy one only obtain a spectrum of the light that actually passes through the slit(all other light is lost). However, the spatial information is crucial to study the outflow, such as the sphere of influence, kinematic energy and momentum. These information help to understand how feedback influence evolution of host galaxy, furthermore, how it change the large-scale kinematics environment. The spatial resolved emission(absorption)-line profile may provide information about how feedback change the chemical environment of ICM and CGM and constrain the theoretical models. So Integral field spectroscopy(IFS) observation is a powerful tool for studying outflows. \cite{harrison2014kiloparsec} used integral field unit(IFU) observations covering OIII$\lambda \lambda 4959,5007$ and H$\beta$ emission lines to study the outflows of 16 type-2 AGN at z<0.2, they found high-velocity ionized gas with observed spatial extents of 6-16kpc. their study demonstrates that galaxy-wide energetic outflows are not confined to the most extreme star-forming galaxies or radio-luminous AGN. \cite{rupke2019100} shows optical integral field observations of the low-redshift galaxy SDSS J211824.06+001729.4 with KCWI, the OII lines at wavelengths of 3726 and 3729 angstroms reveal an ionized outflow spanning 80 by 100 square kiloparsecs. 

Comparing with low-redshift observation, observation for high-redshift outflow is insufficient in contrast to dramatic galaxy evolution at this epoch. Several pieces of observational evidence suggest that the most dramatic galaxy evolution occurred at high redshift($z \approx 1-3$). Firstly, the cosmic space density of BH growth peaks at $z \approx 1-2$ while over half of the integrated BH growth occurred around these redshift(\cite{schmidt1983quasar};\cite{richards2006sloan}). Secondly, the peak density of the most intensely star-forming galaxies is at $z \approx 2$(\cite{chapman2005redshift};\cite{wardlow2011laboca}), with the peak in cosmic star formation density occurring at the same stage(\cite{madau1996high};\cite{lilly1999canada};\cite{madau2014cosmic}). Studies of local galaxies have also suggested that massive galaxies formed the bulk of their stars at high redshift($z \geqslant 1-2$ ). In conclusion, this is an essential stage for galaxy evolution, outflow observation of this epoch is crucial to reveal how galaxies co-evolve with their environments. 

Motivated by the above reasons, in this paper we report a follow-up observation on the famous overdensity field centering on BOSS1441 with Keck Cosmic Web Imager(KCWI). This is an extremely overdense region at $z \approx 2.3$ known as MAMMOTH-1 which was found by \cite{cai2015mapping}. He shows that BOSS1441 has an Ly$\alpha$ emitter(LAE) obverdensity $\delta \approx 10$. Also, the Mayall-4m telescope NB imaging shows that an extreme ly$\alpha$ nebulae with projected size$\geqslant$ 400kpc covers this region(\cite{cai2017discovery}). All of this shows this is a very interesting field at high redshift. In Section 2 we give details of IFU observations. In Section 3 we present the results and model used to estimate outflow energy and in Section 4 we discuss the ly$\alpha$ nebula and the possible mechanism to power the strong outflow. Finally we give a brief summary in Section 5. Throughout this paper, we assume a flat cosmological model with $\omega_{\Lambda}=0.7$,$\omega_{m}=0.3$ and $H_{0}=70 km/s/Mpc$. In this cosmology, 1"$ \approx 8.2$kpc at z=2.3.



















%In this paper we present Keck Cosmic Web Imager(KCWI) blue sensitive Integral Field Spectrometer(IFS) observation of MAMMOTH-1 which is the denest region ever seen at $z \approx 2$. In Section 2 we give details of IFU observations, In Section 3 we present the results and model used to estimate outflow energy. Finally we discuss the ly$\alpha$ nebula and the possible mechanism to power the strong outflow, and give a brief summary in Section 5. Throughout this paper, we assume a flat cosmological model with $\omega_{\Lambda}=0.7$,$\omega_{m}=0.3$ and $H_{0}=70 km/s/Mpc$. In this cosmology, 1"$ \approx 8.2$kpc at z=2.3.
\end{document}