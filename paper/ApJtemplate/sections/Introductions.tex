\documentclass[../main.tex]{subfiles}

\begin{document}
In the local universe, most galaxies are found massive cluster with 
a giant elliptical galaxy locating at center. The current hierarchical 
structure formation model predicts that these central galaxies merge 
with several nearby satellite galaxies to build up their stellar mass. 
This violent merging process is thought to take place in the highest 
density peaks in the early Universe, the so-called protoclusters. 
There are several decades year efforts on searching for these structures 
at high-redshift and some methods have been proposed. However it is still 
under debates that on which systems represent the nurseries of present-day 
massive clusters. 

Previous work searching for protocluster base on high-redshift radio galaxies(HzRGs). This method is supported by ly$\alpha$ emitters(LAEs) overdensities near them, and in some cases by overdensities in submillimeter observations. Being the host of active galactic nucleus(AGN) and characterized by intense radio emission, HzRGs are also known for their associated giant ly$\alpha$ nebulae on hundreds of kpc scales. However, \cite{cai2015mapping} proposed a new method, Coherently Strong intergalactic ly$\alpha$ Absorption systems(CoSLAs), to search for overdensities. With this approach he discovered an extremely massive overdensity BOSS1441 at z=2.32. Through narrow band observation with Large Binocular Telescope(LBT),\cite{cai2017discovery} found an enormous ly$\alpha$ nebulae(ELANe) projected distance of which is beyond 400kpc. This is also the largest ELANes ever discovered. In the cosmic hierarchical nature of structure formation, large-scale filaments are formed out of the merging of small-scale pieces. Simulations suggest that cosmic webs containing baryonic matter tend to align with underlying large-scale structures of dark matter. So, the discovery of such large scale nebulae is a strong evidence of cosmic large-scale structure and the hierarchical model of galaxies formation.

Futhermore, the long-slit observations indicate the ELANe of MAMMOTH-1 has a drastically different kinematics compared to previously discovered ELANes. It is the first radio-quiet nebulae that is associated with strongly extended HeII CIV OIII and H$\alpha$ emission. The emission lines can be fit by two major components about 1000km/s. This result suggests there is strong feedback powered by central AGN. AGN are fascinating objects, the huge amounts of energy released by the active super massive black hole(SMBH) can have an impact on the life and evolution of their entire host galaxy. This process is called AGN feedback, which is a key ingredient in the theory model of galaxies evolution. AGN feedback is now included in many theoretical, numerical and semi-analytic models. The goal of the observation is to provide constraints to help a realistic implementation of feedback effects in simulations. several evidences show powerful outflow from AGN. \cite{sturm2011massive} used Herschel-PACS find high velocities above 1000 km/s and mass outflow rate $\dot{M} \approx 1200 M_{\odot} /yr$ in Ultra-Luminous Infared Galaxies(ULIRGs). There're also many reports of outflows from galaxies hosting AGN 1000 km/s outflows have been seen either side of an obscured quasar at low redshifts. Besides, strong outflows at high redshift are also observed. In most cases outflows are discovered by metal absorption lines such as OIII or CII. This method is popular to for outflows search because there's huge quantities of material in galaxies and we can estimate its peculiar velocity by spectra. A spectacular example is the 1300 km/s outflow in a quasar at $z \approx 6.4$ revealed by broad wings of CII emission line(\cite{cicone2015very}). The kinetic power in the outflow is $\dot{E} \approx 2 \times 10^{45} erg/s$. 

In this paper we present Keck Cosmic Web Imager(KCWI) blue sensitive Integral Field Spectrometer(IFS) observation of MAMMOTH-1 which is the denest region ever seen at $z \approx 2$. In section 2 we give details of IFU observations, In section 3 we present the results and model used to estimate outflow energy. Finally we discuss the ly$\alpha$ nebula and the possible mechanism to power the strong outflow, and give a brief summary in section 5. Throughout this paper, we assume a flat cosmological model with $\omega_{\Lambda}=0.7$,$\omega_{m}=0.3$ and $H_{0}=70 km/s/Mpc$. In this cosmology, 1"$ \approx 8.2$kpc at z=2.3.
\end{document}