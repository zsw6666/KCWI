\documentclass[../main.tex]{subfiles}

\begin{document}
Over the past few decades it has been realized that the black hole (BH) at the centre of a galaxy bulge is more than just decorations but play a key role in galaxy evolution \citep{fabian4114observational}. 
This process is known as Active Galactic Nucleus (AGN) feedback which take place by accreting matter onto the massive BH and have a profound effect on galaxy evolution even the large-scale structure comparing with the large ratio of the size of BH to host galaxy.
%The process by which this occurs is known as AGN(Active Galactic Nucleus) feedback and it takes place through an interaction between the energy and radiation generated by accretion onto the massive BH and the gas in the host galaxy. In contrast to the large ratio of the size of BH to host galaxy, AGN have a potent effect on galaxy formation and evolution even the large-scale structure. 
For example, the energy released to build a BH with mass $\rm M_{BH}=10^{8} \mathbf{M}_{\odot}$ would correspond to $\rm E_{BH} \approx 0.1M_{BH}c^{2}$. This total accretion energy is two-to-three order of magnitude higher than the binding energy of the galaxy bulge the BH reside in, and is comparable to, even higher than, the thermal energy of the gas in the dark matter halo \citep{harrison2016observational}. As a consequence, feedback induced by massive BH has the ability to change the evolution path of galaxies. Observations are required to constrain the details of when, where and how these processes actually occur.
%So the study on AGN feedback is extremely important for galaxy even cosmology.

%Theoretical models of galaxy formation and evolution have found it necessary to implement AGN feedback processes in order to produce the observational evidence. 
There are two sorts of AGN feedback mechanisms. The first one is the radio mode feedback also known as maintenance mode or kinetic mode, it is believed that this mode is the most efficient mechanism in low redshift clusters, at late times and during periods of low BH accretion rates. The objects that are responsible for this type of feedback are likely to be low-excitation radio AGN which has low accretion rate and are always companied by radio lobes \citep{churazov2005supermassive,bower2006breaking,mccarthy2011gas}. This sort of feedback is mainly used to explain the cut-off of the galaxy luminosity function in cluster environment \citep{bower2006breaking,croton2006many,somerville2008semi}. In addition, a more powerful form of interaction between AGN and its environment happens during periods of rapid accretion known as quasar mode or radiative mode. Objects which is responsible for this type of feedback are the radiatively-efficient AGN possessing large accretion rate. Theoretical models for this form of feedback typically require $\rm \approx 0.005-0.15$ of the accretion energy to couple to the cold gas within the host galaxy and to expel this gas through outflows which ultimately results in the shut-down of future BH growth or star formation \citep{benson2003shapes,hopkins2006normalization,debuhr2012galaxy}. Besides, analytical models have used the idea of galaxy-scale outflows initially launched by AGN, to explain $\rm M_{BH}-M_{bulge}$ relationship \citep{fabian1999obscured,granato2004physical,king2011large,faucher2012physics}. This form of feedback is thought to be effective at producing high mass outflow rates and quenching star formation. In addition to the potential affects of AGN feedback described above, AGN-driven outflows are required to blow the gas in their host galaxy out to explain the chemical enrichment of intercluster medium (ICM) and circumgalactic medium(CGM) \citep{borgani2008chemical,wiersma2009chemical,fabjan2010simulating,ciotti2010feedback}. In contrast, it has been prompted that, these outflows could induce positive feedback which trigger star formation by inducing pressure in cold gas reservoirs \citep{nayakshin2012quasar,ishibashi2012active,silk2013unleashing}. However, there are still many fundamental questions to be solved, it is not well established how the accretion energy couples to the gas to drive outflows and the impact of outflow on a much larger physical scale. 

%It is believed that AGN activity is required to drive the highest velocity outflows and are particularly important for the evolution of most massive galaxies(\citep{benson2003shapes};\citep{mccarthy2011gas};\citep{harrison2018agn}). 
%AGN-driven outflows are initially launched from the accretion disk or dusty torus surrounding the BH. 


Spectroscopy is used as a direct way to search for outflows. Such observations have identified outflows in ionized, atomic and molecular gas \citep{nesvadba2008evidence}. A method that is widely used to search for outflowing ionized gas is the high-velocity OIII$\rm \lambda5007$ emission line. This is a good tracer of kinematics in the narrow-line region (NLR) of AGN and people have used it to study kinematics in hundreds to ten of thousands of AGN \citep{wang2011evolution,mullaney2013narrow} to constrain the ubiquity of these outflow features and study them as a function of key AGN properties. 
However, These studies have used only one-dimensional spectra and therefore provide no insight on the spatial extent or structure of the outflows. Even for long-slit spectroscopy only spectrum of light passing through the slit is obtained. In contrast, the spatial information is crucial to study the outflow, such as the sphere of influence, kinematic energy and momentum. These kinds of information help us to understand how feedback influence the evolution of host galaxy, furthermore, how it change the large-scale kinematics environment. The spatial resolved emission (absorption)-line profile may provide information about how feedback change the chemical environment of ICM and CGM and constrain the theoretical models. Hence, Integral field spectroscopy (IFS) is required for these studies. \citet{harrison2014kiloparsec} used integral field unit (IFU) observations covering OIII$\rm \lambda \lambda 4959,5007$ and H$\rm \beta$ emission lines to study the outflows of 16 type-2 AGN at z<0.2. They found high-velocity ionized gas with observed spatial extents of 6-16 kpc. Their study demonstrates that galaxy-wide energetic outflows are not confined to the most extreme star-forming galaxies or radio-luminous AGN. In another observation, \citet{rupke2019100} shows optical integral field observations of the low-redshift galaxy SDSS J211824.06+001729.4 with Keck Cosmic Web Imager (KCWI), the OII lines at wavelengths of 3726 and 3729 angstroms reveal an ionized outflow spanning 80 by 100 square kiloparsecs. 

Comparing with low-redshift observation, observation for high-redshift outflow is insufficient in contrast to how dramatic galaxy evolution at this epoch. Several pieces of observational evidence suggest that the most violent galaxy evolution occurred at high redshift ($\rm z \approx 1-3$). Firstly, the cosmic space density of BH growth peaks at $\rm z \approx 1-2$ while over half of the integrated BH growth occurred around these redshift \citep{schmidt1983quasar,richards2006sloan}. Secondly, the peak density of the most intensely star-forming galaxies is at $\rm z \approx 2$ \citep{chapman2005redshift,wardlow2011laboca}, with the peak in cosmic star formation density occurring at the same stage \citep{madau1996high,lilly1999canada,madau2014cosmic}. Studies of local galaxies have also suggested that massive galaxies formed the bulk of their stars at high redshift ($\rm z \geqslant 1-2$ ). In conclusion, this is an essential stage for galaxy evolution, observation of the universal outflow at this epoch is crucial to reveal how galaxies co-evolve with their environments. 

Motivated by the above reasons, in this paper we report a follow-up observation on the famous overdensity field centering on BOSS1441 with KCWI. This is a cluster core at $\rm z \approx 2.3$ known as MAMMOTH-1 which was found in \citet{cai2017discovery}. It is shown that BOSS1441 has an Ly$\rm \alpha$ emitter (LAE) obverdensity $\rm \delta \approx 10$. The follow-up narrow band (NB) imaging with Mayall-4 m telescope reveals a widely extended enormous Ly$\rm \alpha$ nebula (ELAN) with projected scale $\rm \geqslant 400 \ kpc$ around BOSS1441 \citep{cai2017discovery}. These works reveal that a large reservoir of cool Ly$\rm \alpha$ emitting gas can exist in the core of a cluster at z>2. Nevertheless, the warm gas content ($\rm 10^{5}$ K) in the z>2 cluster environment is not probed yet. In contrast to observation, The nature of the powering mechanism of this nebula is still unclear.
%All of this shows this is a very interesting field at high redshift. 
In Section 2 we give details of IFU observations and data preprocessing. In Section 3 we present the results and model used to estimate outflow energy In Section 4 we discuss the ly$\rm \alpha$ nebula and the possible mechanism to power the strong outflow. Finally we give a brief summary in Section 5. Throughout this paper, we assume a flat cosmological model with $\rm \omega_{\Lambda}=0.7$,$\omega_{m}=0.3$ and $\rm H_{0}=70 km/s/Mpc$. In this cosmology, 1"$\rm \approx 8.2$ kpc at z=2.3.


\end{document}