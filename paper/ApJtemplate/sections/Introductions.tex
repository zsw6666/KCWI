\documentclass[../main.tex]{subfiles}

\begin{document}
Over the past few decades it has been realized that the black hole (BH) at the centre of a galaxy bulge is more than just decorations but plays a key role in galaxy evolution \citep{fabian4114observational}. This process is known as Active Galactic Nucleus (AGN) feedback which takes place by accreting matter onto the massive BH and have a profound effect on galaxy evolution even the large-scale structure comparing with the large ratio of the size of BH to host galaxy. For example, the energy released to build a BH with mass $\rm M_{BH}=10^{8}  M_{\odot}$ would correspond to $\rm E_{BH} \approx 0.1M_{BH}c^{2}$. This total accretion energy is two-to-three orders of magnitude higher than the binding energy of the galaxy bulge the BH resides in, and is comparable to, even higher than, the thermal energy of the gas in the dark matter halo \citep{harrison2016observational}. So feedback induced by massive BH is able to change the evolution path of galaxies and observations are required to constrain details of this process.


After decades of studies, two sorts of AGN feedback models have been developed. The first one is the radio mode feedback also known as maintenance mode or kinetic mode, it is believed that this mode is the most efficient mechanism in low redshift clusters, at late times and during periods of low BH accretion rates. The objects that are responsible for this type of feedback are likely to be low-excitation radio AGN which has low accretion rate and are always companied by radio lobes \citep{churazov2005supermassive,bower2006breaking,mccarthy2011gas}. This feedback model can naturally solves the "cut-off" of the galaxy luminosity function, cooling flow problem and so on \citep{bower2006breaking,croton2006many,somerville2008semi}. In addition, another powerful mode of feedback happens during periods of rapid accretion known as quasar mode or radiative mode. AGN with high accretion rate is responsible for this type of feedback. This mode basically requires 0.005-0.15 of the accretion energy to couple to the cold gas within the host galaxy and to expel this gas through outflows. This ultimately results in the shut-down of future BH growth or star formation \citep{benson2003shapes,hopkins2006normalization,debuhr2012galaxy}. Nevertheless, it has been also prompted that, this mode of feedback could induce positive feedback which trigger star formation by inducing pressure in cold gas reservoirs \citep{nayakshin2012quasar,ishibashi2012active,silk2013unleashing}. Analytical models have already used it to explain $\rm M_{BH}-M_{bulge}$ relationship \citep{fabian1999obscured,granato2004physical,king2011large,faucher2012physics} and this mode is also required to explain the chemical enrichment of intercluster medium (ICM) and circumgalactic medium CGM) \citep{borgani2008chemical,wiersma2009chemical,fabjan2010simulating,ciotti2010feedback}. 


Spectroscopy is used as a direct way to search for outflows. Such observations have identified outflows in ionized, atomic and molecular gas \citep{nesvadba2008evidence}. The high-velocity [OIII]$\rm \lambda5007$ emission line is widely used to search for outflowing ionized gas. These studies have used only one-dimensional spectra and therefore provide no insight on the spatial extent or structure of the outflow. Nevertheless, the spatial information is crucial to study the outflow, such as the three-dimensional kinematics. This information help us to further understand how feedback influences the coevolution of the host galaxies with the gas environment and also provide us information about the metallicity of CGM and ICM. \citet{harrison2014kiloparsec} used integral field unit (IFU) observations covering [OIII]$\rm \lambda \lambda 4959,5007$ and H$\rm \beta$ emission lines to study the outflows of 16 type-2 AGN at $\rm z<0.2$. They found high-velocity ionized gas with observed spatial extents of 6-16 kpc. Their study demonstrates that galaxy-wide energetic outflows are not confined to the most extreme star-forming galaxies or radio-luminous AGN. In another observations, \citet{rupke2019100} show optical integral field observations of the low-redshift galaxy SDSS J211824.06+001729.4 with Keck Cosmic Web Imager (KCWI), the OII$\rm \lambda \lambda 3726,3729$ emission lines reveal an ionized outflow powered by starburst spanning 80 by 100 square kpc. 

At the redshift of $\rm z=1-3$, the peak period of cosmic star formation, the jets have even been discovered on scale of 100 kpc by observing either the radio lobes or x-ray cavities. Meanwhile, some studies also suggest quasars have the ability to ionize and accelerate dense clumps of material on sub-kpc scales while quasar feedbacks are even invoked on larger scales. Previous "Quasars Probing Quasars" (QPQ) survey have found signatures of quasar-mode feedback in the cool CGM through absorptions studies \citep{Prochaska_2013,Hennawi_2013,Lau_2016}. For example, \citet{Lau_2016} used CIV absorptions to trace the velocity field in CGM. They found, on 100 kpc, the velocity dispersion of CIV can still reaches to $\rm \sim 500 \ km/s$ which may indicate intense activities. These results suggest quasar-mode feedback has the ability to significantly affect the CGM of quasars. In theoretical aspect, many works use the approach of hydrodynamical numerical simulation to study the AGN feedback in galaxy formation and evolution \citep{Springel_2005,Di_Matteo_2005,Ciotti_2007,Sijacki_2007,Booth_2009,Ostriker_2010,Hirschmann_2014,Ciotti_2017}. Although some simulations \citep{Faucher_Gigu_re_2016} are able to reproduce the high covering fraction of optically thick gas, their velocity fields are not extreme enough.

Nevertheless, the QPQ survey and other absorption-line studies are limited by the relatively small sample size of bright, background sources which is hardly to provide more than one sightline passing through any halo. To resolve the spatial distribution, global kinematics and key properties of the gas, emission studies of medium are needed. Owing to its diffuse nature, such observations are generally extremely difficult, but the newly on-board KCWI, an integral field spectroscopy are able to t push surface brightness limits down to a few $\rm \times 10^{-19} erg \ s^{-1} \ cm^{2} \ arcsec^{-2}$ and better. This make it possible to directly probe the metal-enriched CGM through emission. In this premise, we report the follow-up observations of \citet{cai2017discovery} on the ELAN with KCWI. This is a newly-onboard integral field spectrograph equipped on Keck Telescope which provides us the ability to capture the very weak photons from the gas in cosmic web. As a result, by using KCWI we reach a deep surface brightness of $\rm 2.2 \times 10^{-18} erg \ s^{-1} \ cm^{2} \ arcsec^{-2}$ and completely reveal the three-dimensional kinematics of Ly$\rm \alpha$ on the scale of 175 kpc (limited by instrument FoV) which is the full KCWI field of view. Other than Ly$\rm \alpha$, the data also presents extended HeII and CIV emission up to 90 kpc. The three-dimensional kinematics extracted from these emissions further confirm the extreme quasar-mode feedback powered by BOSS1441. The paper is structured as follows, in \S 2 we give details of IFU observations and data preprocessing. In \S 3 we present the optimal-extracted images and moment maps of these emission lines. Models used to estimate outflow energy are also presented in this section. In \S 4 we discuss the possible mechanisms powering Ly$\rm \alpha$ nebula the outflow. Finally we give a brief summary in \S 5. Throughout this paper, we assume a flat cosmological model with $\rm \Omega_{\Lambda}=0.7$, $\Omega_{M}=0.3$ and $\rm H_{0}=70 \ km \ s^{-1} \ Mpc^{-1}$. In this cosmology, 1"$\rm \approx 8.2$ kpc at $\rm z=2.3$.


\end{document}