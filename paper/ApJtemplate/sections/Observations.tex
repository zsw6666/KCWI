\documentclass[../main.tex]{subfiles}

\begin{document}
	In this section, we provide details on the KCWI instrument configurations, observations,data reduction pipeline, and post-processing after standard reduction pipeline
\subsection{KCWI Instrument Configuration}
Keck Cosmic Web Imager(KCWI) is a general purpose, optical IFS that has been installed on the 10m Keck II telescope. KCWI provides seeing-limited imaging from the wavelength of $3500 \AA - 5700 \AA$, and the spectral resolution can be configured from R=1000 to R=20000. The field of view is $20'' \times 33''$ for large slicer, $16'' \times 20''$ for medium slicer and $8'' \times 20''$ for the small slicer. KCWI is optimal for a survey of gasous nebulae at $z \approx 2$ because: (1) KCWI has a high throughput from $3800 \AA - 4500 \AA$,optimal for probing ly$\alpha$ CIV and HeII lines at $z \approx 2$. (2) KCWI has a high spectral-resolution modes(R>4000) which can resolve the gas kinematics and (3) KCWI has a relatively large field-of-view(FoV) to cover extended ly$\alpha$ nebulae. Futhermore, KCWI nicely complements the characteristics of MUSE which thrives at $\lambda> 5000 \AA$. Data was taken with the Keck/KCWI instrument in November 2017. The seeing varied in the range of 0.7-1.1 arcsec.
\end{document}