\documentclass[../Results.tex]{subfiles}

\begin{document}
	Deep VLA observation on MAMMOTH-1 with 2.3" resolution (see in \cite{emonts2019cold}) shows radio CO(1-0) emission which trace the cold molecular gas (T$\approx$10-100 K) on the central source of MAMMOTH-1(we use the terminology in \cite{cai2017discovery} and call it source-B in following sections) and its companions. He used D-configuration of VLA and centered on the frequency of 34.808 GHz($\nu_{rest}$=115.27 GHz). After 14 hrs exposure, the results shows extended peak radio flux which has ~ 5 kpc offset from the stellar body of source-B (~30 kpc beyond 3$\sigma$). Besides, he also detected 3 faint radio sources in the same region. With CO(1-0) emission he calculated the total mass of molecular gas $\mathrm{M}_{\mathrm{H} 2}=1.4\left(\alpha_{\mathrm{CO}} / 3,6\right) \times 10^{11} \mathrm{M}_{\odot}$, furthermore, he found an unusually narrow velocity dispersion(FWHM~85 km/s), follows diffuse light seen in HST imaging, he interprets that this molecular CGM in the core likely originated from gas cooling in an enriched multi-phase medium.
	
	From the results of \cite{emonts2019cold}, we doesn't get any confident detection of continuum radio source at the position of Source-B. However, we give an upper limit on its submillimeter emission, $f_{\nu}<0.01 mJy$ at $\lambda_{obs}=8500 \mu m$. But from the results of \cite{arrigoni2018qso}(Fig. 11), this upper limit still cannot rule out the the radio-loud case. But neither the VLA nor ALMA observations detected two distinct radio sources around Source-B, we conclude that Source-B is most likely radio-quiet source.
\end{document}