\documentclass{subfiles}

\begin{document}
	\citet{Cantalupo_2017} demonstrates there are three physical processes that are able to produce extended and bright Ly$\rm \alpha$ emission. The first is the recombination radiation following hydrogen photoionization, the second is Ly$\rm \alpha$ scattering of photons produced by a nearby star forming galaxy or quasar, and the last is Ly$\rm \alpha$ collisional excitation and recombination radiation also known as shock heating. \citet{Leibler_2018} shows that the expected $\rm \frac{F_{Ly\alpha}}{F_{H\alpha}}$ ratio for the "case B" of photoionization recombination should range between 8.1-11.6. The mean ratio for "region 1" is 0.314 and 6.8 for "region 2". This shows H$\rm \alpha$ emission is much weaker than Ly$\rm \alpha$ emission in "region 1" which indicates "case B" of fluorescence is unlikely to be the dominant powering mechanism. In addition, due to the powerful outflow discovered in this nebula, this enormous Ly$\alpha$ nebula is likely to be powered by the shock heating. Because source-B is a starburst quasar confirmed by the SED fitting, it should have strong UV emissions to power the fluorescence. Moreover, due to the decline of ratio from edge to center, we note that shock heating and fluorescence may both contribute to the Ly$\rm \alpha$ emission. Shock heating dominant the emission in the central region while fluorescence may play a role in peripheral region.
\end{document}