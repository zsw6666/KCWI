%% Template for a preprint Letter or Article for submission
%% to the journal Nature.
%% Written by Peter Czoschke, 26 February 2004
%%

\documentclass{nature}

%% make sure you have the nature.cls and naturemag.bst files where
%% LaTeX can find them

\bibliographystyle{naturemag}

\title{An extreme quasar-mode feedback on 100 kpc scale at z=2.3 in an enormous Ly$\rm \alpha$ nebula}

%% Notice placement of commas and superscripts and use of &
%% in the author list

\author{}
%Aauthor$^{1,2}$, Bauthor$^2$ \& LastAuthor$^2$

\begin{document}

\maketitle

\begin{affiliations}
% \item Put institutions in this environment and
% \item separate with \verb|\item| commands.
aa
\end{affiliations}

\begin{abstract}
The details of how galaxy feedback heat and enrich the gas of 
circumgalactic medium are still widely debated (fabian 2012).
Previous simulations and observations suggest that radiation-
driven feedback, also known as quasar-mode feedback launched 
from the disk of super massive black hole in galaxy center 
may play an important role in this process (cite: QPQ, simulations). 
It is believed that when the black hole is under the rapid growth 
episode, a large fraction of accretion energy will be transferred 
to photons and lead to a strong radiation pressure. It can 
drive a very fast wind and have the potential ability to influence 
the larges-scale environment (cite: UFO). Recently, a enormous 
Ly$\rm \alpha$ nebula was discovered in a protocluster locating 
in the densest knot of cosmic web at $\rm z=2.3$ (cite: cai2017, 
fab2018, emonts2019). Unfortunately, imaging and spectra is 
not sufficient to constrain its physical characteristics 
and to reveal its kinematics. Here we report an integral field 
unit observations on this nebula. We find a outflow extending on
large scale in the brightest emission region with a sharp velocity 
gradient and large velocity dispersion. This is the first to directly
see the quasar-mode feedback on circumgalactic galactic medium.

%For Nature, the abstract is really an introductory paragraph set
%in bold type.  This paragraph must be ``fully referenced'' and
%less than 180 words for Letters.  This is the thing that is
%supposed to be aimed at people from other disciplines and is
%arguably the most important part to getting your paper past the
%editors.  End this paragraph with a sentence like ``Here we
%show...'' or something similar.
\end{abstract}

We conducted the observations for MAMMOTH-1 at UT-20180518 with the Keck Cosmic Web Imager (KCWI) which is an integral field spectrograph (IFU) designed for low surface brightness measurements. The image slicer used possesses field of view (FoV) of $\rm 20\arcsec \times 30\arcsec$ and spatial resolution of $\rm 1.35\arcsec$. For these observations, the wavelength coverage of the spectrograph is 3500-5600 \AA \ with slit-width-limited resolution $\rm \Delta \lambda=0.5$ \AA. This is an excellent regime to perform studies of Ly$\rm \alpha$ nebula at $\rm z=2.3$. The method and details of our observations and data reduction are discussed in the Methods section. After the reduction, we obtain a smoothed data cube of dimensions right ascension, declination and wavelength (RA, dec., $\rm \lambda$). It reveals beside Ly$\rm \alpha$ emission, HeII and CIV emission also extend on tens of kpc (Fig. 1). The gradient shown on the three flux-weighted velocity map indicates a powerful outflow from the central source (known as source-B). 

Fig.1 shows the Ly$\rm \alpha$ emission extends to distance of 175 kpc north and south of source-B, and 134 kpc east and west of it while the HeII emission extends to 80 kpc north and south of it and CIV emission extends to 96 kpc north and south of it. The flux centers among these emissions have the same offset of $\rm \sim 8 \ kpc$ from source-B. This indicates the point spread function of source-B has little contribution to the results and the three emissions have the same powering mechanism. This nebula is remarkable comparing to other Ly$\rm \alpha$ not only due to the very luminous and extended Ly$\rm \alpha$ nebula (442 kpc, cite cai2017), but also the unprecedentedly extended HeII and CIV nebula. This results make it the most unique system at $\rm z\sim 2$. Besides, the luminosity of Ly$\rm \alpha$, HeII and CIV is xxx, xxx, xxx respectively. We then calculate the flux ratio between HeII, CIV and Ly$\rm \alpha$ to be $\rm \frac{HeII}{Ly\alpha}=xxx$ and $\rm \frac{CIV}{Ly\alpha}=xxx$. Compare it with simulations, we find the the metallicity of this nebula is $\rm Z\sim xxx Z_{\odot}$. It suggests this nebula is extremely metal-enriched comparing to previous observations on circumgalactic medium (CGM) (cite CGM observations). The unusual nature of this nebula implies perhaps a giant galactic wind ejecting from source-B because if the wind is powerful enough, it is possible to illumine the gas in (CGM as well as enriching the metal environment.

On the basis of its emission and metallicity, the largely extended Ly$\rm \alpha$, HeII and CIV emissions suggests a strong wind emerging from source-B. The spatially resolved gas kinematics confirm this impression and exhibit the two gas components possessing unexpected high velocity on large scale. Ly$\rm \alpha$ emission traces the kinematics of cool gas ($\rm 10^{4} \ K$). The cool gas exhibits the velocity gradient on large scale with the offset of $\rm \Delta v_{Ly\alpha}=xxx \ km/s$. Note that this offset is the low limit because the velocity component parallel to sky is ignored. The channel maps (Fig. 2) confirm this result by the spatially separated red and blue components. The dispersion map also exhibits that this region has a relatively larger dispersion, $\rm \sigma_{v}>400 \ km/s$, with the maximum value of $\rm \sigma_{v}\sim 700 \ km/s$. With the host dark halo mass of $\rm M_{DM}\sim 10^{13}M_{\odot}$ (cite cai2017 discovery of BOSS1441), the upper-limit dispersion of gas due to halo gravity is $\rm \sigma_{up}=xxx \ km/s$ which is much smaller than our result. The high velocity together with the large dispersion imply this wind is extremely powerful with a large open angle, because it can not only lead to the unexpected large velocity gradient on such a large scale, but also give rise to violent turbulence revealed by the dispersion. Note that the Ly$\rm \alpha$ resonant scattering effect can be ignored here because it tends to disrupt the coherency of kinematics instead of enhancing it (cite: cantalupo 2005). 

Furthermore, the HeII and CIV emissions tracing the hot gas ($\rm 10^{5} \ K$) forcefully confirms this argument. It shows the hot gas has the velocity gradient at the same region along the same direction with the cool gas. The velocity offsets are $\rm \Delta v_{CIV}=xxx \ km/s$ and $\rm \Delta v_{HeII}=xxx \ km/s$ respectively, also consistent with the cool gas. The surprising similarity between cool gas and hot gas on the one hand confirms the wind, on the other hand suggests the wind is multiphase. This wind perhaps launched from the inner region of source-B and Fully coupled with metal-enriched hot gas in the galaxy. Due to its high initial velocity, it continues to propagate into the CGM and couples to the cool gas. 

 Studying the spectral energy distribution (SED) of its host galaxy (source-B) can help us to further understand its characters. Previous study (cite fab2018) reveals that the template of M82 fits surprisingly well with the optical and sub-millimeter data fo source-B. So the analysis thus suggests source-B is an strong starbursting galaxy with star formation rate (SFR) of $\rm SFR=400 \ M_{\odot} \ yr^{-1}$. Since no radio data is available, this fitting result is not particularly credible. Fortunately, our observations with the Atacama Large Millimeter Array (ALMA) and the Very Large Array (VLA) make up for this incompleteness. However, our broadband images reveal no statistically significant continuum radio signal on source-B. Therefore, we use the $\rm 3\sigma$ upper limits to constrain the SED of source-B. The results show the upper limits agree well with the fitting template and this confirms the  analysis results. 

Due to the particularity of this high-velocity, metal-enriched galactic wind, the powering mechanism would be extraordinary powerful and efficient. Generally, three models of feedback have been prompted to explain the galactic outflow (cite: stellar feedback; jet feedback; radiation feedback), the maintenance-mode feedback, quasar-mode feedback and star-forming-driven feedback. The first two feedbacks are both caused by the accretion of supper massive black hole but with different efficiency. The maintenance-mode feedback is accompanied by the collimating jet which would further lead to high radio emission (cite radio feedback) while the quasar-mode feedback resides in the very bright quasar which powers the outflow by the strong radiation pressure (cite: quasar-mode feedback). Nevertheless, the star-forming-driven feedback is caused by the star-formation activities, such as stellar winds and supernovae (cite star-forming-driven feedback). 

Based on these facts, we investigate the likely driver of this outflow through the SED of source-B and simulations. By comparison, we find our fitting SED is two-orders-of-magnitude lower than that of radio galaxy in radio band and four-orders-of-magnitude lower than that of "spiderweb galaxy" which is a famous galaxy with powerful jet. This indicates source-B is unlikely to possess strong radio emission due to jet. To quantitatively confirm this argument, we calculate the "radio-excess ratio", $\rm q_{IR}$, which is used to identify if there is radio emission significantly above that expected from the ongoing star formation (cite harrison2014). Our calculation shows $\rm q_{IR}=1.9$ higher than the threshold below which sources are thought to have significant radio emission. This analysis rule out the radio-mode feedback. Besides, in spite of the large SFR, our simulation on star-forming-driven feedback shows wind powering the high-velocity outflow on such large scale requires the initial velocity to be $\rm \sim 2300 \ km/s$ which is impossible based on previous studies (cite star-forming-driven feedback). So the quasar-mode feedback is the only possible mechanism to supply this energetic outflow.

Some estimations on kinematics are required to confirm this and further complete its portrait. Our estimation reveals this 
% By comparing the fitting template with that of radio galaxies, we find its radio flux is order-of-magnitude smaller than that of radio galaxies. 
%Therefore, the $\rm 3\sigma$ upper limits are used to further constrain the SED. It shows these two upper limits also agree well with the template of M82. 
%we use the $\rm 3\sigma$ upper limit as the constrain for the SED. It shows the two upper limits also agree well with the fitting template 
%and this further strongly constraints this SED. Besides, in the radio band our fitting template shows two-order-of-magnitude lower than that of radio galaxies (cite: radio galaxy). 

%Generally, studying the powering mechanism of outflow can help us to understand the physical processes in galaxies, such as the star formation history, the growth rate of the super massive black hole and the bulge-black hole relation (cite), not to mention this extreme case. Three powering mechanisms have been prompted by previous theoretical studies (cite sb-driven, jet-driven, radiation-driven). 
%One common approach to speculate on it is by fitting the spectral energy distribution (SED) of the host galaxy. As a result, the template of M82 which is a starburst galaxy fits surprisingly well with the data of source-B (cite: fab2018). 


%Then the body of the main text appears after the intro paragraph.
%Figure environments can be left in place in the document.
%\verb|\includegraphics| commands are ignored since Nature wants
%the figures sent as separate files and the captions are
%automatically moved to the end of the document (they are printed
%out with the \verb|\end{document}| command. However, tables must
%be manually moved to the end of the document, after the addendum.

\begin{figure}
\caption{Each figure legend should begin with a brief title for
the whole figure and continue with a short description of each
panel and the symbols used. For contributions with methods
sections, legends should not contain any details of methods, or
exceed 100 words (fewer than 500 words in total for the whole
paper). In contributions without methods sections, legends should
be fewer than 300 words (800 words or fewer in total for the whole
paper).}
\end{figure}

\section*{Another Section}

Sections can only be used in Articles.  Contributions should be
organized in the sequence: title, text, methods, references,
Supplementary Information line (if any), acknowledgements,
interest declaration, corresponding author line, tables, figure
legends.

Spelling must be British English (Oxford English Dictionary)

In addition, a cover letter needs to be written with the
following:
\begin{enumerate}
 \item A 100 word or less summary indicating on scientific grounds
why the paper should be considered for a wide-ranging journal like
\textsl{Nature} instead of a more narrowly focussed journal.
 \item A 100 word or less summary aimed at a non-scientific audience,
written at the level of a national newspaper.  It may be used for
\textsl{Nature}'s press release or other general publicity.
 \item The cover letter should state clearly what is included as the
submission, including number of figures, supporting manuscripts
and any Supplementary Information (specifying number of items and
format).
 \item The cover letter should also state the number of
words of text in the paper; the number of figures and parts of
figures (for example, 4 figures, comprising 16 separate panels in
total); a rough estimate of the desired final size of figures in
terms of number of pages; and a full current postal address,
telephone and fax numbers, and current e-mail address.
\end{enumerate}

See \textsl{Nature}'s website
(\texttt{http://www.nature.com/nature/submit/gta/index.html}) for
complete submission guidelines.

\begin{methods}
Put methods in here.  If you are going to subsection it, use
\verb|\subsection| commands.  Methods section should be less than
800 words and if it is less than 200 words, it can be incorporated
into the main text.

\subsection{Method subsection.}

Here is a description of a specific method used.  Note that the
subsection heading ends with a full stop (period) and that the
command is \verb|\subsection{}| not \verb|\subsection*{}|.

\end{methods}

%% Put the bibliography here, most people will use BiBTeX in
%% which case the environment below should be replaced with
%% the \bibliography{} command.

\begin{thebibliography}{1}
\bibitem{dummy} Articles are restricted to 50 references, Letters
to 30.
\bibitem{dummyb} No compound references -- only one source per
reference.
\end{thebibliography}


%% Here is the endmatter stuff: Supplementary Info, etc.
%% Use \item's to separate, default label is "Acknowledgements"

\begin{addendum}
 \item Put acknowledgements here.
 \item[Competing Interests] The authors declare that they have no
competing financial interests.
 \item[Correspondence] Correspondence and requests for materials
should be addressed to A.B.C.~(email: myaddress@nowhere.edu).
\end{addendum}

%%
%% TABLES
%%
%% If there are any tables, put them here.
%%

\end{document}